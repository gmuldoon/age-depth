\documentclass[]{article}
\usepackage{amsmath} %for help with the bracketed conditional equations
\usepackage{graphicx}
\usepackage{multirow}
\usepackage{mathtools}
\usepackage{chngcntr}
\setkeys{Gin}{draft=false}
\begin{document}

\noindent
Attachment XX
\\

To study the past dynamics and configuration of the unstable West Antarctic Ice Sheet, we simulate englacial isochronous reflectors to compare to those observed using ice-penetrating radar. Such isochrones can encode information about past flow direction and speed of the ice sheet which may be different from today. We compare isochrones simulated for the steady state case (i.e. no net mass loss of the ice sheet) to observed isochrones (for which mass loss has been recorded on the ice sheet) to identify areas that may be especially susceptible to ice mass loss. We hypothesize areas where the simulated and observed isochrones differ most are the most sensitive to subglacial boundary conditions or modern climactic influences on the ice sheet. The figure below shows the proof of concept of such a comparison being conducted for Muldoon et al. \textit{in prep}. The left panel shows simulated elevation of an 17,500 year old isochrone and observed elevation to the same isochrone is shown on the right. Higher-resolution simulations and interpolation schemes are currently underway to refine the simulations for direct comparison to multiple observed isochrones.

\begin{figure*}[h]
\begin{center}
\includegraphics[scale=0.5]{model_data_layers.png}
\end{center}
\end{figure*}



\end{document}

