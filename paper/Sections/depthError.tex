


\subsection{Radar depth and error model}
 Radar pulses transmitted into the ice sheet reflect off surfaces of dielectric contrast in the ice that are the result of variations in ice fabric, composition, temperature, and rheology of ice \citep{fujita2000}. The reflected signal is received by the radar system and recorded as two-way travel time (TWTT) from transmission to receipt.  The ice-penetrating radar in which reflectors in this analysis were observed (Figure~\ref{fig:layergram}) was collected by the University of Texas Institute for Geophysics, including GIMBLE \citep{gimble2017}, AGASEA \citep{holt2006}, CASERTZ \citep{morse2002}, and SOAR/WMB \citep{luyendyk2003} (Figure~\ref{fig:radarmap}). Data used to trace reflectors between Byrd ice core and WAIS Divide ice core was collected from a DC-3 or Twin Otter airborne platform and uses the HiCARS coherent radar system with 60 MHz center frequency and 15 MHz bandwidth \citep{peters2005}.% or the TUD-derived incoherent radar sounder \citep{blankenship2001} with 4 MHz bandwidth.

\begin{figure}[h]
\centering
\makebox[\textwidth][c]{
\includegraphics[scale=0.5]{/Users/gail/Documents/Research/Projects/Layers/analysis/figures/WAIS_vel_gray_final}}
%\captionsetup{width=.9\textwidth}
\caption{Map of central West Antarctic with available airborne geophysical radar surveys (yellow lines) and  WAIS Divide and Byrd ice core locations (blue triangles) overlain. Gray shading is surface velocity \citep{rignot2011}. The red line denotes the flight line along which the reflectors in this study were observed. }
\label{fig:radarmap}
\end{figure}



In this study, we consider TWTT of four reflectors spanning the ice thickness in the region of central West Antarctica (Figure~\ref{fig:layergram}). These reflectors have been tracked using Halliburton's Landmark seismic interpretation software and can be tied to both the Byrd and WAIS Divide ice cores for dating using observations from the HiCARS system (Figure~\ref{fig:radarmap}). %Determining the depth of the reflectors, $D_r$, may be affected by several sources of uncertainty, including uncertainty in the firn depth, the velocity of the radar pulse in ice, and the pulse-limited precision of radar observations.

\begin{figure}[h]
\centering
\makebox[\textwidth][c]{\includegraphics[scale=0.6]{/Users/gail/Documents/Research/Projects/Layers/paper/figures/radargram-4layers-paper}}
%\captionsetup{width=.9\textwidth}
\caption{Radargram showing reflectors of interest near the Byrd ice core along flight line ICP6/MKB2l/F14T01a observed using the HiCARS2 radar system. Short vertical hatches along tracked reflectors show intersections with crosslines.}
\label{fig:layergram}
\end{figure}


To estimate reflector depths from TWTT ($TWTT_{m,r}(D_r)$), we use a simple relation between the different velocities of the radar signal in air and in ice and incorporate several known sources of uncertainty, including: 1) variations in the radar velocity in ice due to ice temperature and fabric ($v_{ice}$), 2) vertical precision limitations of radar range detection ($\epsilon_{prec}$), and 3) uncertainty in the firn correction ($d_{firn}$) due to measurement errors in the ice density profile:

\begin{equation}\label{deptheqn}
%D_r= \frac{1}{2}[(TWTT_r + \epsilon_{prec,r}) - (TWTT_{surf} + \epsilon_{prec,surf})] \cdot v_{ice} + d_{firn}\\
TWTT_{m,r}(D_r) + \epsilon_{prec,r} = 2 \frac{D_r - (d_{firn}+\epsilon_{firn})}{v_{ice}} + (TWTT_{surf} + \epsilon_{prec,surf})
\end{equation}

The complexity of local ice properties affecting the velocity at any location and depth make it difficult to know the true velocity. Empirical evidence suggests a range of expected velocities and we conservatively assume they are uniformly distributed such that: $p(v_{ice}) ~\sim U[168,169.5]\,m/{\mu}s$ \citep{fujita2000}. In lieu of detailed observations of ice properties with depth, we assume the value of $v_{ice}$ is a constant throughout the ice column and apply it systematically to all reflector depths.

The radar pulse width determines vertical precision, $\sigma_r$ \citep{millar1982}. We assume a finite pulse width, meaning an infinitesimally thin layer of ice will appear in the survey to have a finite width. This can lead to errors in tracing isochronous reflectors, as the reflected energy from a finite depth will include ice with a range of ages. To account for this, we include uncertainty due to range precision error, $\epsilon_{prec}$, according to the signal to noise ratio of each reflector's radar amplitude as in \citet{cavitte2016}. We assume this error is normal such that $p(\epsilon_{prec}) = N(0,\sigma_r)\,ns$. Values of $\epsilon_{prec}$ are sampled for each reflector independently.

Finally, a firn layer with variable density \citep{gow1970} exists in the upper part of the ice sheet. The velocity of the radar is faster in firn than in solid ice. To correct for the underestimation of ice depth if the firn layer is not considered, we estimate the firn correction ($d_{firn}$), the difference between the ice thickness with and without the firn layer present . %Based on the density profile and firn thickness ($\sim$ 65 m) at the Byrd ice core site \citep{gow1970}, we estimate the prior on $d_{firn}$ to be $p(d_{firn})\sim$ U[6, 36]\,m. This assumes firn density ranges from 400 - 8030 $\frac{kg}{m^3}$ and glacial ice ranges from 830 - 917 $\frac{kg}{m^3}$ \citep{paterson1994}. The firn correction is applied to all estimates of englacial reflector depth in this study because they are all deeper than the measured firn layer.
Errors in density, $\rho(z)$, are used to estimate the error in $d_{firn}$, $\epsilon_{firn}$. These errors are known for the WAIS Divide measurements, but not for the Byrd ice core profile. In lieu of density measurement errors at the Byrd site, we assume the errors to be consistent with those observed at the WAIS Divide ice core. These errors are assumed gaussian, randomly sampled, and the firn correction is computed using the \citet{dowdeswell2004} relation:

\begin{equation}
d_{firn} = \frac{K}{n^{'}_{i}}\int{(\rho_{i} - \rho(z)) dz}
\end{equation}
where K is 0.85 m$^{3}$ Mg$^{-1}$ \citep{robin1969}, $n^{'}_{i}$ is the refractive index of ice ($n^{'}_{i}$=1.78), $\rho_{i}$ is the density of ice ($\rho_{i}$=0.917 Mg m$^{-3}$) and $\rho(z)$ is the density of ice at depth \textit{z} with units Mg m$^{-3}$.

The TWTT from the observing aircraft to the surface of the ice sheet is known from interpretation of the surface reflector, $TWTT_{surf}$. The computed depth of each reflector is relative to this surface reflector. Just as each reflector may have TWTT precision errors independent of the others, errors in the distance between the surface and the acquisition aircraft are common to all observed reflectors in the ice column. Therefore, a randomly sampled precision error, $\epsilon_{prec,surf}$, is applied systematically to all reflectors.







