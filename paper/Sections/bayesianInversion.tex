
% An age profile with associated errors is available at the WAIS Divide ice core site (Buizert, personal communication 2016), however the  site lacks a robust measure of uncertainty. To derive Byrd ice core, we use volcanic events  detected in the Byrd ice core using the electrical conductivity method \citep{hammer1997}. We assume a 3\% uncertainty on the volcanic ages as a rule of thumb (personal communication, T.J. Fudge). We then invert for age given layer depth using a Bayesian MCMC approach and check our result by correlation to the nearby WAIS Divide ice core chronology.

% Our approach uses a Metropolis algorithm to fit an ice flow model to the observed volcanic profile from the Byrd ice core. An ensemble of well-fitting ice flow models is generated from this process which samples uncertainty in age given uncertainty in depth. We then sample from this ensemble to construct a distribution of likely ages for layers of interest. This method allows us to build a model using data from the last 50,000 years which can be used to estimate uncertainties in the chronology at earlier times.

In order to assign ages to observed radar reflectors, we are interested in combining information from radar, ice flow physics, and dates from ice core volcanics at the Byrd ice core site. To do so, we take advantage of the versatility of a Bayesian approach to assemble the desired solution from a set of inter-related components.  Our method preserves the chronologic superposition of the ice column and correlation of errors with depth, estimates the probability of ice flow parameter values, and estimates a joint probability of the age-depth profile. 

%because while some parameters are well known, others are highly uncertain, such as the accumulation function at this site over time. By iteratively inverting for non-unique solutions, we find a distribution of sets of parameter values which are consistent with observations. Specifically, we derive an ensemble of age-depth profiles for the Byrd ice core constrained by agreement with a chronology of volcanic events from the past 50,000 years detected in the Byrd ice core record using electrical conductivity \citep{hammer1997} and with TWTT to bright, continuous reflectors observed in ice-penetrating radar \citep{holt2006,young2012,morse2002} and traced using Landmark seismic interpretation software produced by Halliburton.


%\textbf{MOVE We assume a base level of 1\% uncertainty on the volcanic ages and include a precision parameter, \textit{S}, to quantify additional uncertainty. We invert for age given depth using a Bayesian Markov Chain Monte Carlo approach and check our result by correlation to the nearby WAIS Divide ice core chronology.}

The Bayesian formulation of this problem is:
\begin{equation}\label{eqn:bigproblem}
\begin{split} %allow line break
%\begin{flalign}
\begin{aligned}
PPD(A_r)  \sim  PPD(D_r,\vec{f},d_{firn},v_{ice},S | TWTT_r,A_{IC},D_{IC})  \propto & \\
        p(TWTT_r | D_r,d_{firn},v_{ice}) \cdot p(A_{IC},D_{IC} | \vec{f},S)  \cdot p(D_r) \cdot p(\vec{f}) &\cdot p(S) \cdot p(d_{firn})\cdot p(v_{ice})
%\end{flalign}
\end{aligned}
\end{split}
\end{equation}

Computing the posterior joint probability distribution of reflector ages, $PPD(A_r)$ in Equation~\ref{eqn:bigproblem}, requires jointly estimating the depths of the englacial reflectors of interest ($D_r$), ice flow model parameters and accumulation rate history ($\vec{f}$), firn depth correction ($d_{firn}$), radar velocity in ice ($v_{ice}$), and precision of the Byrd ice core chronology ($S$). We estimate these values given information about observed radar reflector two-way travel time ($TWTT_r$) and the ages and depths of volcanics ($A_{IC}$, $D_{IC}$)interpreted from the Byrd ice core record \citep{hammer1997}.


We use priors, the rightmost five factors in Equation~\ref{eqn:bigproblem}, to put physical bounds on quantities of interest as described in the following sections. Priors on ${D_r}$ also preserve stratigraphic dependence of radar reflectors, requiring deeper reflectors be older than shallower reflectors. Likelihood functions, the first two factors on the right-hand side of Equation~\ref{eqn:bigproblem}, evaluate if ice flow and radar model estimates agree with observations. The age likelihood (second factor in Equation~\ref{eqn:bigproblem}) fits sampled ice flow parameters and accumulation rate history ($\vec{f}$) to estimate an age-depth profile consistent with the volcanic ages and depths ($A_{IC}$, $D_{IC}$). The scatter remaining after this fit determines precision ($S$) and whether estimates of $D_r$, $d_{firn}$, and $v_{ice}$ are rejected or accepted. The TWTT likelihood (first factor in Equation~\ref{eqn:bigproblem}) tests %Sampling first the agreement between flow model results ($\vec{f}$) and the volcanic record ($A_{IC}$, $D_{IC}$) helps to determine precision, $S$, which is given by the scatter between published ages and ice flow model ages.  Secondly, 
agreement between observed two-way travel time ($TWTT_r$) and modeled reflector depths ($D_r$) using information about $d_{firn}$ and $v_{ice}$. Parameter estimates must pass both likelihood tests to be accepted in our solution. The elements of Equation~\ref{eqn:bigproblem} are discussed more thoroughly in subsequent sections. %The supplement provides a more detailed description of how the sequence of sampling and acceptance/rejection was implemented.

%Accepted solutions are expected to satisfy all of these conditions simultaneously, so we take the product of their probabilities in Equation~\ref{eqn:bigproblem}. The Metropolis algorithm \citep{metropolis1953} is used to sample combinations of parameters and boundary conditions consistent with the problem as described. To do so, sets of parameter values for $D_r$, $\vec{f}$, $S$, $d_{firn}$, and $v_{ice}$ are selected according to the Gibbs-Hastings method which steps through parameter space using a proposal based on the previous accepted parameter set, i.e. a markov chain.  (See supplement.) %This method allows us to build a model using data from the last 50,000 years which can be used to estimate uncertainties in the chronology at earlier times.
%Likelihood functions for age and TWTT are evaluated to accept or reject these proposed parameter sets based on the change in distance between the proposed solution and observations compared to the previous step. Solutions are those parameter sets which are the most consistent with observations given the physical limits defined by the priors, observational uncertainty, and scatter between the model and observations. We use 0,000 iterations to estimate the solution to Equation~\ref{eqn:bigproblem}. The resulting ensemble of parameter sets are used to generate an ensemble of age-depth profiles. The resulting posterior probability distribution, $p(D_r, \vec{f},S | TWTT_r, A_{IC}, D_{IC})$, samples the age uncertainty in the Byrd ice core in proportion to the combined probabilities and the age of observed radar reflectors, $p(A_r)$, using the flow model described below.


\subsection{Ice flow model at the Byrd ice core}
%This is consistent with experiment opinion that ice core age uncertainty is generally 2-3$\%$ of ice age (T.J. Fudge, private communication). Therefore, we construct an age-depth distribution for the Byrd ice core using statistical methods and an assumed 3$\%$ age uncertainty. The derived age-depth profile is then compared to the WAIS Divide profile using englacial layers that have been tracked between the two ice core sites. 

Due to the inherent stratigraphic dependence of age in the ice column and the nonlinear effect of ice deformation on layer depths, we use a flow model to simulate the age-depth relationship. We use a simple, one-dimensional model of ice flow (Equation \ref{schwander}) which derives ice age from accumulation and strain rate, assuming constant horizontal strain rate in the upper part of the ice sheet and a shear layer of thickness $h$ at the base of the ice sheet \citep{schwander2001}. In the shear layer, the strain rate is assumed to decrease linearly and the bottom of the ice is assumed to slide with velocity $q$ $\cdot$ $v_0$, where $v_0$ is the horizontal velocity at the surface. The age of ice as a function of elevation from the bedrock interface is therefore:

\begin{equation}\label{schwander}
A(z) = \int_{z}^{H} \frac{dz}{\epsilon_z \cdot \dot{a}(z)}
\end{equation}

\noindent where strain is described by \\

\begin{center}
$    \epsilon_z=
	\begin{aligned}
    \begin{cases}
                 1-k(H-z), \:\:  h \leq z \leq H  & (upper) \\
                  kz(q+\frac{1-q}{2h}z), \:\:\: 0 \leq z < h &(lower) \\
    \end{cases}, 
    \end{aligned}
$
\end{center}
\bigskip

\noindent and $k$ is a constant such that $k = \frac{2}{2H - h(1-q)}$. $H$ is ice thickness, which has been observed to be 2164 m at the Byrd ice core site \citep{gow1968} and $z$ is the height above the bed. We invert for the remaining parameters: $h$, the depth to the Dansgaard-Johnsen shear layer \citep{dansgaardjohnsen1969}; $\dot{a}$, the time-dependent accumulation rate; and $q$, the ratio of horizontal velocity at the surface to that at the bed of the ice sheet.

The ice flow model accounts for two primary factors in the age-depth profile: burial as a function of accumulation rate, $\dot{a}$, and thinning as a function of strain, $\epsilon_z$. In the simplest realization, ice deposited at a given time at the ice sheet surface will be found at a depth corresponding to the amount of subsequent accumulation. However, due to strain thinning at depth, ice of a given age will be less deep than would be expected if accumulation alone is considered. 


The priors used for the ice flow parameters are defined by:
%Among the parameters to be estimated are the ice flow parameters $h$, $q$, and $\dot{a}$. These flow parameters are sampled from uniform priors defined in liberal ranges:

\begin{center}
\begin{equation}\label{priors}
p(\vec{f}) = 
\begin{cases}
p(h) \sim U[0, 0.5] \\
p(q) \sim U[0, 1] \\
p(\dot{a}_{i, i=1...10}) \sim U[0.05,0.25]\,m/a\\
\end{cases}	
\end{equation}	
\end{center}

The prior distributions of flow parameters and accumulation rate history, together denoted as $p(\vec{f})$, assume the shear layer is in the bottom half of the ice sheet depth \citep{cuffey2010} and that ice at the bed of the ice sheet is moving no faster than the surface, which would allow for cases of both plug and creep flow. Accumulation rate as a function of depth, $\dot{a}$, is estimated for 10 distinct depth bins spanning the ice thickness at 200 m intervals. This allows for variability of accumulation rate over time. 

\subsection{Radar depth and error model}
 Radar pulses transmitted into the ice sheet reflect off surfaces of dielectric contrast in the ice that are the result of variations in ice fabric, composition, temperature, and rheology of ice \citep{fujita2000}. The reflected signal is received by the radar system and recorded as two-way travel time (TWTT) from transmission to receipt.  The ice-penetrating radar used in this analysis was collected in several surveys by the University of Texas Institute for Geophysics, including GIMBLE \citep{wais2016}, AGASEA \citep{holt2006}, CASERTZ \citep{morse2002}, and SOAR/WMB \citep{luyendyk2003}. Data has been collected from a DC-3 or Twin Otter airborne platform and uses either the HiCARS coherent radar system with 60 MHz center frequency and 15 MHz bandwidth \citep{peters2005} or the TUD-derived incoherent radar sounder \citep{blankenship2001} with 4 MHz bandwidth.


\begin{figure}[h]\label{fig:radarmap}
\centering
\makebox[\textwidth][c]{
\includegraphics[scale=0.5]{../analysis/figures/WAIS_vel_gray_final}}
%\captionsetup{width=.9\textwidth}
\caption{Map of central West Antarctic with available airborne geophysical radar surveys (yellow lines) and  WAIS Divide and Byrd ice core locations (blue triangles) overlain. Gray shading is surface velocity \citep{rignot2011}. The red line denotes the flight line along which the reflectors in this study were observed. }
\end{figure}



In this study, we consider TWTT of four reflectors spanning the ice thickness in the region of central West Antarctica (Figure~\ref{fig:layergram}). These reflectors have been tracked extensively using Halliburton's Landmark seismic interpretation software and can be tied to both the Byrd and WAIS Divide ice cores for dating. %Determining the depth of the reflectors, $D_r$, may be affected by several sources of uncertainty, including uncertainty in the firn depth, the velocity of the radar pulse in ice, and the pulse-limited precision of radar observations.

\begin{figure}[h]\label{fig:layergram}
\centering
\makebox[\textwidth][c]{\includegraphics[scale=0.6]{figures/radargram-4layers-paper}}
%\captionsetup{width=.9\textwidth}
\caption{Radargram showing reflectors of interest near the Byrd ice core along flight line ICP6/MKB2l/F14T01a observed using the HiCARS2 radar system. Short vertical hatches along tracked reflectors show intersections with crosslines.}
\end{figure}


To estimate TWTT associated reflector depths, $TWTT_{m,r}(D_r)$, we begin with a simple relation relating different velocities of the radar signal in air and in ice. We incorporate several known sources of uncertainty, including: 1) variations in the signal velocity in ice due to ice temperature and fabric ($v_{ice}$), 2) vertical precision limitations of range detection ($\epsilon_{prec}$), and 3) measurement errors in the density profile needed to account for the changing signal velocity in the firn ($d_{firn}$):

\begin{equation}\label{deptheqn}
%D_r= \frac{1}{2}[(TWTT_r + \epsilon_{prec,r}) - (TWTT_{surf} + \epsilon_{prec,surf})] \cdot v_{ice} + d_{firn}\\
TWTT_{m,r}(D_r) + \epsilon_{prec,r} = 2 \frac{D_r - d_{firn}}{v_{ice}} + (TWTT_{surf} + \epsilon_{prec,surf})
\end{equation}

The complexity of local ice properties affecting the velocity at any location and depth make it difficult to know the true velocity. Empirical evidence suggests a range of expected velocities and we conservatively assume they are uniformly distributed such that: $p(v_{ice}) ~\sim U[168,169.5]\,m/{\mu}s$ \citep{fujita2000}. In lieu of detailed observations of ice properties with depth, we assume the value of $v_{ice}$ is a constant throughout the ice column and apply it systematically to all reflector depths.

The radar pulse width determines vertical precision \citep{millar1982}. We assume a finite pulse width, meaning an infinitesimally thin layer of ice will appear in the survey to have a finite width. This can lead to errors in tracing reflectors, as the reflected energy from a finite depth will include ice with a range of ages. To account for this, we include uncertainty due to range precision error, $\epsilon_{prec}$, of 14 ns \citep{cavitte2016}. We assume this error is normal such that $p(\epsilon_{prec}) = N(0,14)\,ns$. Values of $\epsilon_{prec}$ are sampled for each reflector independently.

Finally, a firn layer with variable density exists in the upper part of the ice sheet. The velocity of the radar signal is faster than in solid ice. To correct for the underestimation of ice depth if the firn layer is not considered, we estimate the difference between the ice thickness with and without the firn layer present ($d_{firn}$). Based on the density profile and firn thickness ($\sim$ 65 m) at the Byrd ice core site \citep{gow1970}, we estimate the prior on $d_{firn}$ to be $p(d_{firn})\sim$ U[6, 36]\,m. 

*******This assumes the firn correction is less than the firn depth itself because we expect velocity in firn will vary from that in glacial ice by only a few percent \citep{robin1975}.  The firn correction is systematically applied to all englacial reflectors of interest in this study because they are alldeeper than the measured firn layer.%Errors in density, $\rho(z)$, are available for the WAIS Divide measurements. These are assumed gaussian, randomly sampled, and the firn correction is computed using the \citet{dowdeswell2004} relation:

%\begin{equation}
%z_f = \frac{K}{n^{'}_{i}}\int{(\rho_{i} - \rho(z)) dz}
%\end{equation}
%where K is 0.85 m$^{3}$ Mg$^{-1}$ \citep{robin1969}, $n^{'}_{i}$ is the refractive index of ice ($n^{'}_{i}$=1.78), $\rho_{i}$ is the density of ice ($\rho_{i}$=0.917 Mg m$^{-3}$) and $\rho(z)$ is the density of ice at depth \textit{z} with units Mg m$^{-3}$.

The TWTT from the observing aircraft to the surface of the ice sheet is known from interpretation of the surface reflector, $TWTT_{surf}$. The computed depth of each reflector is relative to this surface reflector. Just as each reflector may have TWTT precision errors independent of the others, errors in the distance between the surface and the acquisition aircraft are systematic across all observed reflectors in the ice column. Therefore, a randomly sampled precision error, $\epsilon_{prec,surf}$, is also applied to $TWTT_{surf}$ which flattened to 250 ns during data processing. 







\label{radardepth}
\subsection{Metropolis Algorithm}\label{metrop}
At each iteration, the Metropolis/Gibbs sampling algorithm \citep{metropolis1953} proposes values for parameters of interest (those with priors in Equation~\ref{eqn:bigproblem}). The algorithm accepts or rejects proposed sets of parameter values by comparison between the proposed and previously-accepted likelihood (or ``cost") values. A low cost value represents good agreement between model and observations, reflecting a small model-data misfit. According to the Metropolis/Gibbs algorithm, if the cost associated with proposed parameters is lower than that of the last accepted iteration, the proposed parameter values are accepted. Alternatively, higher cost values may be accepted with a probability determined by the likelihood. Each likelihood function is considered separately and both must represent an adequate solution on a given iteration for the proposed parameters to be accepted.


There are two likelihood functions, describing the model-data misfit between reflector depth and age, respectively:


\begin{equation}\label{eqn:loglikeage}
p(A_{IC} | D_{IC},\vec{f},S) \propto exp[\frac{-S\sum_{j = 60}[A_{IC,j} - A_{m,j}(\vec{f},D_{IC})]^2}{2\sigma_A^2} + r^6]
\end{equation}

\begin{equation}\label{eqn:loglikedepth}
p(TWTT_r | D_r,d_{firn},v_{ice} ) \propto exp[\frac{-\sum_{i=4}[TWTT_{r,i} - TWTT_{m,i}(D_r)]^2}{2\sigma_{TWTT}^2}]
\end{equation}


In the depth likelihood function, $TWTT_m(D_r)$ is based on the relationship between depth and TWTT as in Equation~\ref{deptheqn} and $TWTT_r$ is observed by ice-penetrating radar for each reflector, $i$. To estimate $\sigma_{TWTT}$, we assume a perfect model and compute errors we expect characteristic of the data. This method allows for correlation between depth errors, as expected. More details are discussed in the Supplement.


In the age likelihood function, a regularization term, $r$, is used to penalize large variability in the sampled accumulation rate function input to the ice flow model and $A_m$ comes from solutions to the forward ice flow model. We train our model on $j=61$ volcanic events from \citet{hammer1997}. Importantly, these volcanic ages do not include uncertainty information. To accomodate additional unknown uncertainty in this data, we use a hierarchical Bayes strategy to infer uncertainty in the volcanic record from scatter between our model and the observed data. This information is encoded in the posterior probability distribution of the parameter $PPD(S)$ as described in \citet{jackson&huerta2016}. 

\begin{equation}\label{eqn:S}
PPD(S) = Ga(\frac{k_e}{2}+\alpha, E_m+\beta)
\end{equation}
where 
\begin{equation}
 E_m= \frac{\sum_{j}[A_{IC,j} - A_{m,j}(\vec{f},D_{IC})]^2}{2\sigma_A^2} 
\end{equation}

Parameters $\alpha$ and $\beta$ are assumed to be 1 as in the case for a noninformative gamma prior. The number of degrees of freedom, $k_e$, is assumed to be less than $j$ due to covariation in the age and depth of ice. Because we do not know the value of $k_e$ a priori, we assume $k_e = \frac{j}{2}$ (see Supplement). As with the other parameters in our problem, values of $S$ are proposed for each Metropolis iteration, effectively estimating reflector age uncertainty given the choice of parameter solution for each iteration.


%The Metropolis algorithm randomly samples each of the ice flow parameters and computes an age-depth profile. A Bayesian method is used to determine suitability of proposed ice flow parameters in a Bayesian way:
%\begin{equation}\label{posterior}
% p(\vec{f},\sigma_V | A_V, D_V) \propto p(A_V,D_V | \vec{f},\sigma_V) p(\vec{f}) p(\sigma_V),
%\end{equation}
%A posterior probability, $p(A | \vec{f})$, is computed for each set of parameters from likelihood and prior probabilities, $P(\vec{f})$. As described below, we require only the proportional relation because relative posterior values are considered in this Markov Chain process.

%The simulated age-depth profile is compared to the observed volcanic chronology using the ``log-likelihood" function, $p(A_V,D_V | \vec{f},\sigma_V)$, to quantify how closely a sampled set of parameters represents the observed ice profile:
%
%\begin{equation}\label{cost}
%ln(likelihood) = \frac{1}{N}\sum\frac{(A_{sim}(\vec{f})_i~ - ~A_{V,i})^2}{(2\sigma_{V})^2},
%\end{equation}
%where $\vec{Age}_{sim}$ comes from evaluating the ice flow model for parameter values sampled from $\vec{f}$. $A_{V}$ and $\sigma_{V}$ describe the observed volcanic chronology at the Byrd ice core. The prior on $\sigma_{V}$ is 
%\begin{equation}
%p(\sigma_V) \sim Ga(\alpha, \beta)
%\end{equation}

%which assumes error in the chronology is $\sim$3\% (T.J. Fudge, pers. comm.). Uncertainty in the volcanic chronology, $\sigma_{V}$, loosens the constraint on how closely $A_{sim}$ must match $A_{V}$ to be acceptable. We assume $N$ degrees of freedom, where $N$ is the number of volcanic observations. This is because observed global volcanic events are assumed independent.
%



%\subsubsection{Estimating accumulation}\label{accum}
The accumulation rate, $\dot{a}$, is highly uncertain for the Byrd ice core, so we compare ice flow model solutions using several accumulation functions. These include constant accumulation, a theoretical function derived for an expected atmospheric temperature timeseries at the nearby WAIS Divide ice core\citet{morse2002}, constant accumulation rates every 5,000 years, and constant accumulation rates in four depth bins which correspond to observed changes in layer thickness at the byrd ice core \citep{}. \textbf{Adding an accumulation function based on Byrd ice core isotopes.}

The result is shown in Figure \ref{fig:accumfunc}. We see that \textbf{One of them does better than the others. This one will be used in the final model to give the real result of age-depth for all the reflectors.}


%\begin{figure}
%%\begin{center}
%\centering
%\includegraphics[scale=0.4]{figures/accumfunc}
%%\captionsetup{width=.9\textwidth}
%\caption[]{\textbf{Unfortunately, this figure is missing the 5k year interval which is the best (but takes the longest to run). In general, the accumulation functions fit the volcanic data well, but perhaps too well as seen in the comparison with the WAIS Divide ice core. This is a proof of concept that the data can be modeled using the MCMC method and we therefore can compute an age-depth profile even where there is no data, but the uncertainty so far is being underestimated.}}
%%\end{center}
%\label{fig:accumfunc}
%\end{figure}



