\subsection{Metropolis Algorithm}\label{metrop}
At each iteration, a hybrid of Hastings and Gibbs sampling \citep{hastings1970,gelfand1992} is used to select values for parameters of interest (those with priors in Equation~\ref{eqn:bigproblem}). The algorithm accepts or rejects proposed sets of parameter values by comparison between the proposed and previously-accepted values, as measured by the likelihood. A high likelihood value represents good agreement between model and observations. According to the Hastings algorithm, if the likelihood associated with proposed parameters is higher than that of the previous accepted iteration, the proposed parameter values are accepted. Alternatively, lower likelihood values may be accepted with a probability determined by the change in likelihood.


There are two likelihood functions describing the model-data misfit in reflector depth and age, respectively:

\begin{equation}\label{eqn:loglikedepth}
\begin{split}
p(TWTT_r | D_r,& d_{firn},v_{ice} ) \propto  \\
& exp[\frac{-\sum_{r=4}[TWTT_{r} - TWTT_{m,r}(D_r)]^2}{2\sigma_{TWTT}^2}]
\end{split}
\end{equation}

\begin{equation}\label{eqn:loglikeage}
\begin{split}
p(A_{IC} | D_{IC},&\vec{f},S) \propto \\
& exp[\frac{-S\sum_{j = 61}[A_{IC,j} - A_{m,j}(\vec{f},D_{IC})]^2}{2\sigma_A^2} + R^6]
\end{split}
\end{equation}
Both likelihood functions must lead to acceptance in order for the proposed parameters to be accepted.

In the depth likelihood function, $TWTT_{m,r}(D_r)$ is based on the relationship between estimates of $D_r$ and TWTT as in Equation~\ref{deptheqn}. $TWTT_r$ is observed by ice-penetrating radar for each reflector, $r$. Uncertainty in TWTT, $\sigma_{TWTT}$, is taken to be the same as the radar range precision error which is a function of the signal-to-noise of each reflector amplitude and the bandwidth of the HiCARS radar system as described previously.

%To estimate $\sigma_{TWTT}$, we assume a perfect model and compute with standard deviation of the numerator given typical errors we expect affect our data. This method allows for correlation between depth errors, as expected. More details are discussed in the Supplement.


In the age likelihood function, the modeled age, $A_m$, is a function of ice flow model parameters and accumulation rate history, $\vec{f}$. A regularization term, $R^6$, is used to penalize large variability in the accumulation rates input to the ice flow model. $R$ is a constant for each proposal equal to the ratio of the variance of the smoothed to unsmoothed proposed accumulation function. $A_m$ comes from solutions to the forward ice flow model. We use $j=61$ volcanic events from \citet{hammer1997} as the observational target, $A_{IC}$, which do not include uncertainty information. These data represent dated volcanic deposits observed in the Byrd ice core and extend to $\sim$ 50 ka, though there is a lack of data in the brittle zone of the ice core between 300 and 900 m depth where the electrical conductivity cannot be measured \citep{hammer1997}. Age uncertainty, $\sigma_A$, is nominally taken to be 1\% of reflector age, a presumed underestimation of the true uncertainty. To determine additional uncertainty in volcanic age, we include a precision parameter, $S$, and use it to infer uncertainty in the volcanic record from scatter between our model and the observed data. In Bayesian nomenclature, $S$ is a ``nuisance" parameter that accounts for uncertainty in $\sigma_A$ by using the sum of squared errors, $E_m$, as a measure of scatter between modeled age, $A_m$, and observed volcanic age, $A_{IC}$. 

%To help determine a posterior probability distribution of $p(S) \sim Ga(\alpha,\beta)$ \citep{jackson&huerta2016}. 
The posterior probability distribution of $S$ is:

\begin{equation}\label{eqn:S}
PPD(S) = Ga(\frac{k_e}{2}+\alpha, E_m+\beta)
\end{equation}
where 
\begin{equation}
 E_m= \frac{\sum_{j}[A_{IC,j} - A_{m,j}(\vec{f},D_{IC})]^2}{2\sigma_A^2} 
\end{equation}

Parameters $\alpha$ and $\beta$ are assumed to be 1 as in the case for a noninformative gamma prior, $p(S)\sim Ga(\alpha,\beta)$. Unlike other parameters in our problem, values of $S$ are selected through Gibbs sampling \citep{gelfand1992}, effectively estimating reflector age uncertainty given the choice of ice flow model parameters and accumulation rate history for each iteration.




%The Metropolis algorithm randomly samples each of the ice flow parameters and computes an age-depth profile. A Bayesian method is used to determine suitability of proposed ice flow parameters in a Bayesian way:
%\begin{equation}\label{posterior}
% p(\vec{f},\sigma_V | A_V, D_V) \propto p(A_V,D_V | \vec{f},\sigma_V) p(\vec{f}) p(\sigma_V),
%\end{equation}
%A posterior probability, $p(A | \vec{f})$, is computed for each set of parameters from likelihood and prior probabilities, $P(\vec{f})$. As described below, we require only the proportional relation because relative posterior values are considered in this Markov Chain process.

%The simulated age-depth profile is compared to the observed volcanic chronology using the ``log-likelihood" function, $p(A_V,D_V | \vec{f},\sigma_V)$, to quantify how closely a sampled set of parameters represents the observed ice profile:
%
%\begin{equation}\label{cost}
%ln(likelihood) = \frac{1}{N}\sum\frac{(A_{sim}(\vec{f})_i~ - ~A_{V,i})^2}{(2\sigma_{V})^2},
%\end{equation}
%where $\vec{Age}_{sim}$ comes from evaluating the ice flow model for parameter values sampled from $\vec{f}$. $A_{V}$ and $\sigma_{V}$ describe the observed volcanic chronology at the Byrd ice core. The prior on $\sigma_{V}$ is 
%\begin{equation}
%p(\sigma_V) \sim Ga(\alpha, \beta)
%\end{equation}

%which assumes error in the chronology is $\sim$3\% (T.J. Fudge, pers. comm.). Uncertainty in the volcanic chronology, $\sigma_{V}$, loosens the constraint on how closely $A_{sim}$ must match $A_{V}$ to be acceptable. We assume $N$ degrees of freedom, where $N$ is the number of volcanic observations. This is because observed global volcanic events are assumed independent.
%


