For this study, we employ a simplified ice flow model which does not include the effects of upstream ice flow on the reflector ages and assumes one-dimensional vertical strain and constant horizontal strain rate in the upper part of the ice sheet. Below, the model includes shear layer in which horizontal strain decreases linearly to the bed that is allowed to slide. We do not include a separate term for basal melting, which is likely occurring at this site because liquid water was observed at the bed during drilling \citep{Gow1968}. Despite this simplification, our results agree with two independent ice core chronologies.

The four reflectors in this study were chosen as a representative sample of the ice column and for their brightness and continuity. It is possible to use this method to date additional englacial reflectors observed using radar in this area. However, the usefulness of such dates only extend as far as the isochronous reflectors can be traced. This method is less helpful for dating discontinuous reflectors, though it could inform relative ages for reflectors seen coincident with dated reflectors but which can be traced only over short distances. 

All such dating efforts should agree to within uncertainty at multiple ice core locations as in \citet{cavitte2016}. We see agreement within 2$\sigma$ between our estimated reflector ages at the Byrd ice core and those observed at the WAIS Divide ice core, though the mean of the WAIS Divide site estimates tend younger than those at the Byrd ice core site. It is unclear why this is the case, though it may be attributable to additional missing uncertainties in drilling of the Byrd ice core. Through use of local density measurements at the WAIS Divide ice core, we exclude differences in firn thickness to account for this discrepancy. However, as an early deep-drilling effort, less sophisticated technology used at the Byrd ice core may have led to more deviations in the drilling direction than reported, for example. Deviations from vertical may lead to a longer core sample than the true ice thickness and may skew the reported volcanic chronology (as measured from the ice core) to be older than it really is. 

Our method is sensitive to the possibility that uncertainty in the estimate of different parameters might be correlated. This is consistent with expectations that parameter values will covary with depth. Because there is not independence between parameters, we must make an assumption about the effective number of degrees of freedom for the ice flow parameters in the problem.  For our age likelihood and estimation of $S$, we assume $k_e$ = 50\% of the number of volcanic data points and do not expect this to change the mean estimates of reflector age and depth, as shown in the Supplemental Information. However, the choice of $k_e$ does have an influence on the uncertainty as indicated in Equation~\ref{eqn:S}.\\




