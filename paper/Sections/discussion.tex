For this study, we employ a simple Dansgaard-Johnsen-type ice flow model. We do not include a separate term for basal melting, which is likely occurring at this site as liquid water was observed at the bed during drilling \citep{gow1968}. This method could be adapted to incorporate a more sophisticated flow model with additional parameters in the inversion. However, despite this simplification, our results agree with two independent ice core chronologies.

Our results also demonstrate the importance of our choice of priors, particularly in accumulation rate parameters at depth. We chose uninformative priors due to our lack of knowledge about local accumulation rates at the Byrd ice core site prior to the LGM. However, more informed priors could be assumed to perhaps better constrain accumulation rate estimates at the bottom of the ice column. 

The four reflectors in this study were chosen as a representative sample of the ice column and for their brightness and continuity. It is possible to use this method to date additional englacial reflectors observed using radar in this area. However, the usefulness of such dates only extend as far as the isochronous reflectors can be horizontally traced. This method is less helpful for dating discontinuous reflectors, though it could inform relative ages for sections of discontinuous reflectors adjacent in the ice column to dated reflectors \citep{macgregor2015}.

All such dating efforts should agree to within uncertainty at multiple ice core locations as in \citet{cavitte2016}. We see agreement between our estimated reflector ages at the Byrd ice core and those observed at the WAIS Divide ice core, though uncertainties in our estimates of radar reflector age are notable larger than uncertainties derived from the WAIS Divide ice core alone. These relatively large uncertainties from the radar can be attributed to radar range precision in the determination of reflector depths. This precision is a function of the signal to noise ratio (SNR) of the radar reflection and bandwidth of the radar system used to obtain the reflections \cite{cavitte2016}. Increasing SNR would improve range precision, however we have already selected for high SNR in analyzing only the brightest reflectors in the ice column, so we expect the data do not support improving precision in this way. Increasing bandwidth (15 MHz for the HiCARS radar system used to obtain these reflectors) would also improve the radar range precision, however this is technically challenging for modern radar systems because antenna tuning is difficult for bandwidths greater than 25\% of the center frequency (60 MHz for the HiCARS system). Further, while higher resolution can be obtained by increasing bandwidth and center frequency, penetration through the ice declines and reflections become increasingly discontinuous \citep{cavitte2016}.

Our method is sensitive to correlations between parameter estimates, consistent with expectations that parameters may covary with depth. Because there is not independence between parameters, we must assume the effective number of degrees of freedom, $k_e$, for the ice flow parameters in the problem.  For our age likelihood and estimation of $S$, we assume $k_e$ = 50\% of the number of volcanic data points. As shown in Figure~\ref{fig:ke}, this choice does not affect mean estimates of reflector age and depth, but it does have some affect on the uncertainty associate with them. Our result could therefore be improved through more work toward an independent determination of an appropriate value for $k_e$. 



 %estimates tend younger than those at the Byrd ice core site. It is unclear why this is the case, though it may be attributable to additional missing uncertainties in drilling of the Byrd ice core. Through use of local density measurements at the WAIS Divide ice core, we exclude differences in firn thickness to account for this discrepancy. %However, as an early deep-drilling effort, less sophisticated technology used at the Byrd ice core may have led to more deviations in the drilling direction than reported, for example. Deviations from vertical may lead to a longer core sample than the true ice thickness and may skew the reported volcanic chronology (as measured from the ice core) to be older than it really is. 

