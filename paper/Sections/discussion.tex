For this study, we employ a simplified ice flow model which does not take into account x, y, and z. It utilizes a Dansgaard-Johnsen-derived flow relationship but allows for horizontal velocity at the bed. We do not include a separate term for basal melting, which is likely occurring at this site because liquid water was observed at the bed during drilling. 

The four reflectors in this study were chosen as a representative sample of the ice column and for their brightness and continuity. It is possible to use this method to date additional englacial reflectors observed using radar in this area. However, the usefulness of such dates only extend as far as the isochronous reflectors can be traced. This method is not applicable for dating discontinuous reflectors, though it could inform methods of determining relative dates of reflectors seen coincident with dated reflectors but which can be traced only over short distances. 

All such dating efforts should agree to within uncertainty at multiple ice core locations. We see agreement within 1$\sigma$ between our estimated reflector ages at the Byrd ice core and those observed at the WAIS Divide ice core. Interestingly, the mean of the latter estimates tend younger than those at the Byrd ice core site. It is unclear why this is the case, though it could be attributed to additional uncertainties in drilling of the Byrd ice core. As an early deep-drilling effort, less sophisticated technology may have led to more deviations in the drilling direction than reported, for example. Deviations from vertical may lead to a longer core sample than the true ice thickness and may skew the reported chronology to be older.

Our method is sensitive to the possibility that uncertainty in the estimate of different parameters might be correlated. This is consistent with expectations because we expect parameter values will covary with depth. Because there is not independence between parameters, we must make an assumption about the effective number of degrees of freedom in the problem. 
- may depend on choice of ke (what would it take to calculate? why didn't last method work?)



