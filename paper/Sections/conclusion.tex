

We derive ages for isochronous radar reflectors observed near Byrd ice core which includes estimates of uncertainty in ice flow parameters and ice properties in the region of Marie Byrd Land, West Antarctica, for which additional observations from radar surveys are just becoming available. Such radar observations reveal englacial stratigraphy indicative of past ice flow, but require dating to put constraints on interpretations of ice dynamics. The Byrd ice core is strategically located to connect the WAIS divide to the ice streams of the Siple Coast and the Marie Byrd Land icecap, but its chronology has lacked robust estimates of uncertainty until now. 

In addition to estimating uncertainty in the chronology for the full ice column, we estimate the age and depth of four radar reflectors. Our estimates of the reflector age-depth are confirmed by independent comparison to the same reflectors dated using the WAIS Divide ice core chronology \citep{buizert2015}. 

In the process of updating the ice chronology near the Byrd ice core, we also compute self-consistent estimates (with uncertainty) of local ice characteristics, such as the accumulation rate as a function of depth, the firn correction, and the velocity of EM waves in ice. Our method also samples uncertainty in estimates of depth from radar observations. These parameter estimates can be used to inform ice sheet modeling efforts in this area and for other studies of paleo ice flow through the Last Glacial Maximum. 

Our results indicate the oldest continuous radar reflector dateable using existing ice cores and radar surveys in West Antarctica is located at $\sim$70\% ice depth at the Byrd ice core site and dates to $\sim$26ka. The same reflector is observed at $~\sim$80\% ice depth in the WAIS Divide ice core. While the Byrd ice core has been isotopically dated to as old as $\sim$94ka \citep{blunier2001}, continuous radar reflectors do not extend deep enough in this region to leverage the ice core to date older aspects of the ice geometry in the central WAIS.




