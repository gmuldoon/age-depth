

We derive ages for isochronous radar reflectors observed near the Byrd ice core which include estimates of uncertainty in ice flow parameters and accumulation rates in the region of Marie Byrd Land, West Antarctica. Such radar observations may reveal englacial stratigraphy indicative of past ice flow, but require dating to put constraints on interpretations of ice dynamics. The Byrd ice core location connects WAIS Divide to the ice streams of the Siple Coast and the Marie Byrd Land icecap via radar observations and this work contributes to constraining uncertainty in the chronology of englacial radar reflectors.

Our estimates of the reflector age-depth are consistent with independent comparison to the same reflectors dated using the WAIS Divide ice core chronology \citep{buizert2015}, although large uncertainties due to radar range precision preclude a strong test of inferences between the two cores. Our results indicate the oldest continuous radar reflector dateable using existing ice cores and radar surveys in central West Antarctica is located at $\sim$70\% ice depth at the Byrd ice core site and dates to $\sim$ 24.9 $\pm$ 0.3 ka. The same reflector is observed at $\sim$80\% ice depth in the WAIS Divide ice core. While the Byrd ice core has been dated to as old as $\sim$ 94 ka at its base \citep{blunier2001}, continuous radar reflectors do not extend deep enough in this region to leverage the ice core to date older radar-observed ice in the central WAIS. The range precision of existing ice-penetrating radar systems is the biggest contributor to uncertainty in radar reflector age. This uncertainty is largely irreducible due to practical trade-offs in airborne system design and leads to potentially significant uncertainty in reflector age at depth. 
%The uncertain accumulation rate history at this location also contributes to overal reflector age and depth uncertainty. 
%This uncertainty could perhaps be reduced with more chronologic data at the Byrd ice core, in addition to the volanic chronology we employ. 

%This analysis also includes self-consistent estimates (with uncertainty) of local ice characteristics, such as the accumulation rate as a function of depth, the firn correction, and the velocity of EM waves in ice. These parameter estimates can be used to inform ice sheet modeling efforts in this region and for other studies of paleo ice flow through the LGM. 






