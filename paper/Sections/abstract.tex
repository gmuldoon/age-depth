Englacial radar reflectors in the central West Antarctic Ice Sheet (WAIS) contain information about past dynamics and ice properties. Thanks to significant data coverage in this area, these isochronous reflectors can be traced over large portions of the ice sheet, but assigning ages to the reflectors for the purpose of studying dynamics requires incorporation of chronologic data from ice cores. To date reflectors in the Marie Byrd Land region, we consider the Byrd ice core, strategically located between the catchments of Thwaites Glacier and the Siple Coast ice streams. However, the deep Byrd ice core chronology lacks uncertainty estimates which are important to confidently date far-reaching englacial isochrones. 

Here we determine ages with uncertainty for a series of four englacial radar reflectors spanning the ice thickness using Bayesian approaches to combine data and uncertainty information from radar observations, an existing Byrd ice core chronology, and a simplified ice sheet model. This method returns a posterior representing the probability distribution of depth and age for observed radar reflectors while allowing for correlation of errors between ice sheet parameters with depth. We present a complete age-depth profile estimate with uncertainty for the Byrd ice core which compares favorably with the more recent WAIS Divide ice core record. Our results estimate ice sheet parameters such as accumulation rate history and suggest the deepest continuous radar reflector is approximately 25.42 $\pm$ 2.49 ka, far younger than the estimated age recorded at the bottom of the ice core. 