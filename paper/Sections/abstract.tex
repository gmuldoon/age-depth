Englacial radar reflectors in the central West Antarctic Ice Sheet (WAIS) contain information about past dynamics and ice properties. With significant data coverage in this area, these reflectors -- assumed to be isochronous --  can be traced over large portions of the ice sheet, but assigning ages to the reflectors for the purpose of studying dynamics requires incorporation of chronologic data from ice cores. The Byrd ice core is strategically located between the catchments of Thwaites Glacier and the Siple Coast ice streams and is an important source of chronologic information for the Marie Byrd Land region. However, this deep ice core chronology lacks quantitative uncertainty estimates which are important to confidently date far-reaching englacial isochrones. 

Here we date a series of four englacial radar reflectors spanning the ice thickness using Bayesian approaches to combine data and uncertainty information from radar observations, an existing Byrd ice core chronology, and a simplified ice sheet model. This method allows for correlation errors between ice sheet parameters with depth and returns a posterior representing the probability distribution of depth and age for observed radar reflectors. Our results reveal estimates of ice sheet parameters such as accumulation rate history and suggest the deepest continuous radar reflector is approximately 25.42 $\pm$ 2.49 ka, far younger than the estimated age recorded at the bottom of the ice core. We present a complete age-depth profile estimate with uncertainty for the Byrd ice core which compares favorably with the more recent WAIS Divide ice core record. 