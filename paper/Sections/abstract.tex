Englacial radar reflectors in the central West Antarctic Ice Sheet (WAIS) contain information about past dynamics and ice properties. Due to significant data coverage in this area, these isochronous reflectors can be traced over large portions of the ice sheet, but assigning ages to the reflectors for the purpose of studying dynamics requires incorporation of chronologic data from ice cores. To date reflectors in the Marie Byrd Land region, we consider the Byrd ice core, strategically located between the catchments of Thwaites Glacier and the Siple Coast ice streams. 

Here we determine ages with uncertainty for a series of four englacial radar reflectors spanning the ice thickness using Bayesian approaches to combine radar observations, an existing Byrd ice core chronology, and a simplified ice sheet model. This method returns the joint probability distribution of depth and age for observed radar reflectors. The inferred age-depth profiles at the Byrd ice core site compare favorably with the more recent WAIS Divide ice core record. The results include inferences of accumulation rate history that show an expected decline in accumulation rate during the Last Glacial Maximum. The deepest continuous radar reflector is approximately 25.42 $\pm$ 2.49 ka, far younger than the estimated age recorded at the bottom of the ice core. 