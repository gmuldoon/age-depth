Englacial radar reflectors in the central West Antarctic Ice Sheet contain information about past dynamics and ice properties. Due to significant data coverage in this area, these isochronous reflectors can be traced over large portions of the ice sheet, but assigning ages to the reflectors for the purpose of studying dynamics requires incorporation of chronologic data from ice cores. To date reflectors in the Marie Byrd Land region, we consider the Byrd ice core, strategically located between the catchments of Thwaites Glacier and the Siple Coast ice streams. We determine ages with uncertainty for four englacial radar reflectors spanning the ice thickness using Bayesian approaches to combine radar observations, an existing Byrd ice core chronology, and a simplified ice flow model. This method returns the marginal probability distribution of depth and age for each of the observed radar reflectors. The results also include inferences of accumulation rate at the Byrd ice core site during the last 30ka that show a minimum accumulation rate during the Last Glacial Maximum at half the modern rate. The deepest continuous radar reflector is 25.67 $\pm$ 1.45 ka, less than 30\% of the estimated age of the oldest ice at the Byrd ice core site despite being located at 70\% of the ice depth, limiting the age of radar-interpretable ice in this region. The inferred reflector age profiles at the Byrd ice core site derived here compare favorably with the more recent WAIS Divide ice core record. However, uncertainty in reflector depth due to radar range precision contributes considerably to uncertainty in reflector age in a way that is not readily reducible using currently available ice-penetrating radar systems.