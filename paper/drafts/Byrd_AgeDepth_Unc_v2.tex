\documentclass[draft,jgrga]{agutex}
%\usepackage{lineno}
%\linenumbers*[1]

%\noindent $>$latex file \\
%\noindent $>$dvips -P pdf -o file.ps file.dvi \\
%\noindent $>$ps2pdf13 file.ps file.pdf

%\authorrunninghead{GUTOWSKI ET AL.}
%\titlerunninghead{Radar uncertainty}

\begin{document}

\title{Uncertainty of Dating Ice at the Byrd Ice Core, Antartica}
\author{Gutowski et al.}

\begin{abstract}
Data from radar-sounding surveys of West Antarctica are used to
determine the age of observed radar horizons near the Byrd ice core,
Antarctica \ref{Gow68}. We emphasize inclusion of uncertainty in
radar- and ice-related sources of uncertainty. We present a new
chronology of the Byrd ice core which takes into account theoretical
and measurement uncertainties in depth and age of prominent layers
observed in the ice.
\end{abstract}

\section{Introduction}
%Basin-scale radar sounding surveys offer a breadth of information
%about the history and dynamics of ice sheets. However, this
%information only becomes available with reliable interpretations of
%the age of isochronous layers mapped through the domain. 


Annual layers within ice sheets have long been recognized as an
invaluable resource for understanding changes in ice dynamics over
time. Information imbedded in these layers -- including their chemical
composition and their position -- provide both climate proxies on both
millennial (and longer) and shorter, annually-resolved
scales. Intrinsic to this, is the ability to assign a date to internal
ice layers that allow for a picture of the underlying dynamics to
emerge. Such information is pertinent to science ranging from the 
development of
large-scale ice sheet models being incorporated into modern general
circulation models to understanding fundamental ice physics to
tracking the mass balance of land ice.

Ice cores provide a useful resource for extracting data from annual
layers, particularly young ones, because these layers can often be
differentiated by eye in a core sample. However, the resulting
measurements are inherently localized and depend on the quality and
completeness of the core sample as it is recovered. 

Increasingly, radar echo sounding has provided a method to obtain 
information about ice layers over large areas. These operations can 
be executed more
quickly and less expensively while provided a wealth of valuable
data. By comparing these data to that of nearby ice cores, it is
possible to determine an age-depth relationship for observed
layers. This relationship can then be used to understand how ice has
flowed throught large regions of ice sheets such as Greenland and
Antarctica over the course of thousands of years.

This information is particularly relevant to studies of sea-level
change; by understanding how ice is flowing, it is possible to
constrain mass balance calculations and therefore track ice
melt. Correlating these features to paleoclimate could lead to a
predictive strategy for estimating sea level rise.

Despite the advancement in our understanding of ice physics, however,
there has yet to be a thorough review of the uncertainties associated
with depth estimations of radar echo surveys. While there are several
existing ice core chronologies that incorporate estimations of age
uncertainties, they are largely incomplete. Although underdeveloped,
uncertainty estimations are at the crux of predicting future sea level 
changes. Without proper uncertainty quanitification, predictions are 
generally useless in matters of decision-making. The community is in
need of robust, probabilistic measures of uncertainty pertaining to
questions of sea-level rise.
 
In this case, uncertainties can be assigned to both the depth of
layers and their corresponding ages. There is inherent uncertainty in
both the radar sampling of layers used to determine their depths and
the myriad of techniques used to determine relative ages of ice layers
and then correlate them to absolute chronologies using known climatic
events. In order to determine a complete picture of the uncertainty
in the age-depth relationship in a region and we emphasize a method
for doing so here.


\section{Data}

The radar echo sounding data was obtained in December 2004 as a part
of the AGASEA project. The flight line passed $~$870 m from the Byrd
ice core site. The plane was travelling at 67 m/s and was 550 m above the
ground. It had a slight -0.37 degree roll and was travelling due
east. The data includes radar times (the time it takes for a radar
pulse to leave the plane, reflect off of a layer, and return to the
detectors) in microseconds. Our data was collected with a 15 Mhz
bandwitch. The 10 unique layers we include here were
hand-chosen for having some of the strongest reflection amplitudes.

Volcanic data was used to quantitatively compare our calculated
age-depth relationship to the age of layers in the Byrd ice core. The
volcanic chronology was developed using the electrical conductivity
method \citet{Hammer94}. We used a representative subset of dated
volcanic events to cover an age range from 709 BP to more than 18000 BP. 
These events correspond to a depth range of 97.8 m to 1890 m below the
1968 surface on the Byrd ice core \citet{Gow68}.

Density data was used to account for the varying amount of ice (as
opposed to air) in the upper part of the ice sheet. This was necessary to properly
calculate the depth of each radar-detected layer we used (see
\ref{unc} for more on this). We obtained our data from the
original analysis of the ice core presented by \citet{Gow68}.




\section{Sources of Uncertainty}\label{unc}
There are many sources of uncertainty inherent to the way in which
data is collected, analyzed, and understood. We have included the
following sources of uncertainty.

\subsection{Uncertainty in depth of radar-detected layers}

As mentioned previously, layers are selected by hand using the program
GeoFrame. This program allows a user to easily select strong
reflectors from a radargram and trace them along a flight
path. However, there is a fundamental limit to how accurately the
depth of these layers can be tracked given the resolution of radar
timing and therefore the sampling rate. The sample rate might be
anywhere from 5ns to 20ns, so we assume a resolution of 
10 ns when picking the reflection
from the surface of the ice sheet and for each internal
layer. We can then treat the two cases of surface and internal layers
separately.

To put this uncertainty into units of depth, we scale the time by
the velocity of the electromagnetic wave involved. For example, we
assume the e/m wave travelled to the surface with a velocity of c, the
speed of light, but then slowed to the velocity of electromagnetic
radiation in ice, which we assume to be 1.69 x 10$^8$ m/s (reference?)
before reaching the internal layers. This results in a 1$\sigma$ depth
uncertainty of 0.3 m for the surface and 0.17 m for the internal
layers. We also consider the fact that the uncertainty picking the
surface represents a systematic error -- it will be the same for each
of the internal ice layers. However, the uncertainties of each
internal layer will not necessarily all be the same, so they should be
modeled as random within the bounds described above.

The radar data used for this study was taken using a radar pulse with
a 15 MHz bandwidth. The limitation of a finite bandwidth means that
an even an infinitesimally thin layer of ice will appear in the survey
to have a finite width. To account for this effect, we assume a
1$\sigma$ depth uncertainty of 5.63 m. This is obtained from considering both
the bandwidth frequency and the velocity of electromagnetic radiation
in ice. This uncertainty is applied as a random error for each of our selected
layers. See Section \ref{method} for additional discussion about how
all of the sources of uncertainty were included.



\subsection{Uncertainty in determination of age}\label{ageunc}

Each year fresh snow accumulates on the top of the ice sheet,
burying the previous season's snowfall. As layers of ice descend into
the ice sheet, the layers become thinner, as air from the surface is
squeezed out and gravity compacts the ice. This thinning makes it
increasingly difficult to distinguish one layer from another at
depth while in shallow regions, it maybe be possible to simply count
layers by eye and therefore determine the age of those layers. 

The uncertainty associated with determining ages for ice layers is a
function of depth;
-delta age \\
-landmarks like 10Be, CH4, F \\
-ecm method accuracy \\
-correlation with bc89? \\

\section{Method}\label{method}
-firn offset \\
-depth correction from 1968  \\
1. determination of depth from radar times \\
2. use metropolis algorithm to invert for accumlation and ratio of
surface to basal velocity based on volcanic dating (calculate cost
based on this) \\
3. use schwander model to calculate ages (include assumptions)\\


We used an ice model adapted from \citet{Schwander01} to determine the
age of internal ice layers near the Byrd ice core drilling site in
Antarctica. The age was determined assuming an average time-varying
accumulation, ice flow, and layer depth near the ice core. We
separated this process into two distinct parts.

First, we use a metropolis algorithm to assign an age to each meter of
depth at the Byrd ice core. We invert for a piecewise accumulation
function over the depth of the core and for a parameter that describes
to the ratio of surface ice velocity to bed ice velocity. We include
the age uncertainties described in Section~\ref{ageunc}. By
comparing to the age and depth of known volcanic events, determined
using the electrical conductivity method (ECM; \citet{Hammer94}), we





Next, we utilized a radar uncertainty model to determine depth from
available radar reflection horizons near the Byrd ice core. Our basic
approach calculated distance based on the speed of light in ice and
the time it took for the radar pulses to return to leave and return to
the aircraft. This method assumes that there is no basal melting and
that the density throughout the ice depth is constant. For simplicity,
we assume throughout this work that there is no basal melting, though
we know that is not the case \citep{Gow68}. We correct for varying
density in the firn layer using a correction presented by
\citet{Dowdeswell04}. Ice density was tracked throughout the ice by
\citet{Gow68}, and we apply these measurements to the Dowdeswell et
al. correction:

%\equation{Dowdeswell} 

Below the firn layer (64 m, ref:Physics of Glaciers), a constant
correction of 6.9 m is applied.



\section{Results}

\section{Discussion}

\section{Conclusion}

\section{References}

\bibliographystyle{apj}
\bibliography{layers}
\end{document}
