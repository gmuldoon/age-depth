

\documentclass[draft,jgrga]{agutex}
%\usepackage{lineno}
%\linenumbers*[1]

%\noindent $>$latex file \\
%\noindent $>$dvips -P pdf -o file.ps file.dvi \\
%\noindent $>$ps2pdf13 file.ps file.pdf

%\authorrunninghead{GUTOWSKI ET AL.}
%\titlerunninghead{Radar uncertainty}

\begin{document}

\title{Interpretation and Uncertainty of Byrd ice core data and
airborne radar sounding observations} \author{Gail Gutowski}

\section{Abstract [ 200 words]}
Data from radar-sounding surveys of West Antarctica are used to
determine the age of observed radar horizons near the Byrd ice core,
Antarctica \citep{Gow68}. We emphasize inclusion of uncertainty in
radar- and ice-related sources of uncertainty. We present a new
chronology of the Byrd ice core which takes into account theoretical
and measurement uncertainties in depth and age of prominent layers
observed in the ice.
%\end{abstract}


\section{Introduction [ 4 pages double spaced]}
The West Antarctic Ice Sheet (WAIS) has become an important focus of
	glaciological research because it is the last marine ice sheet
	on Earth. A full 75$\%$ of the ice sheet is based below sea
	level, leaving the region susceptible to the Marine Ice Sheet
	Instability Problem. [source] This problem is characterized by
	the unique bedrock topography of West Antarctica, in which the
	bedrock slopes down toward the interior of the continent. As
	the ice sheet retreats due to melting or calving at the front,
	the grounding line moves back to a thicker part of the ice
	sheet. In the absence of longitudinal forces or a buttressing
	ice shelf to hold back the ice sheet, discharge will increase,
	leading to further retreat. This process acts as a positive
	feedback to the system, resulting in accelerated retreat of
	the grounding line until the ice sheet reaches a steady state
	position upstream.

Marine ice sheet instability could
	therefore be responsible for a rapid disintegration of up to
	75$\%$ of the West Antarctic ice sheet. This scenario would
	result in a global sea level rise approaching 5 m. [source]
	Such a severe rise in sea level would threaten tens of
	millions of individuals worldwide living in Low Elevation
	Coastal Regions. [source?] It is therefore important to gain a
	better understanding of ice flow in West Antarctica in order
	to gauge its stability in response to climate change.

	Isochronous layers within glaciers provide one avenue for
	developing an understanding of ice dynamics.  Each winter,
	snow is deposited onto the surface of a glacier. This snow
	contains signatures of atmospheric conditions and other
	isotopic clues about climatic conditions at the time of
	deposition. [source] Over time, each layer is covered with the
	accumulation of subsequent years, where it is compacted first
	into a layer of firn and then into ice as it becomes buried in
	the ice column. This process produced isochronous layering
	within the glacier that carries a signature of the surface
	conditions at the time of snowfall.   

One such signature is
	the accumulation rate.  The thickness of an isochronous layer
	represents the accumulation over the period of deposition of
	that layer. As the layer is advecting down into the glacier,
	it will thin as a result of gravitational compaction, but it
	is possible to correct for this effect and so retain some
	information about accumulation rate is the period of
	deposition can be discerned. Accumulation rates and patterns
	of accumulation are related to several climatic factors of
	interest including air temperature and surface
	topography. [source] This makes it possible to begin
	recreating ancient climates from long-buried layers of ice. 
	
It is also possible to learn about ice flow from drawdown and
	deformation observed in the layers. Layer draw down, in which
	layers move deeper in the glacier (with or without a
	corresponding change in bed topography) may be indicative of
	basal melting. [e.g. source] In some cases, observation reveal
	layers disappearing from the bottom of the ice sheet, a sign
	that ice is melting from the base. This could correspond to
	areas of fast-moving ice (past or present) such as an ice
	stream, or may indicate an area of increased geothermal
	heating.  

%Further disruption of ice layers may indicate
%	even more complex ice dynamics in places within the
%	glacier. [source?] Areas of disruption may warrant additional
%	observations to shed light on processes of ice flow we have
%	yet to understand.  

In addition to their usefulness for
	understanding ice dynamics, these isochronous ice layers
	provide a glimpse into the interior of a glacier where other
	observational methods are rendered useless. Ice-penetrating
	radar surveys of ice sheets reveal internal layers as radar
	reflection horizons. These horizons can be tracked over large
	glacial regions, in some cases tracing ice flow across
	hundreds of kilometers, as in the case of West
	Antarctica. [source] As mentioned earlier, this allows for
	studies of paleoclimate over large regions of an ice sheet. In
	East Antarctica, it presents scientists with the opportunity
	to compare ice core analyses between stations located
	throughout the region. [source] This cross-correlation
	provides an unprecedented opportunity for uncertainty
	quantification of ice core chronologies. 

Because internal layering encodes information about accumulation rates
	and deformation as ice flows, they can be used to develop a
	picture of ice dynamics. One of the main obstacles in ice
	sheet modeling is the uncertainty surrounding basal boundary
	conditions. One way to explore this problem is to use surface
	observations to invert for basal parameters. [source]
	Dynamical information about the flow of ice within the column,
	derived from analyses of internal isochrones can further
	inform these inversion problems to more accurately describe
	and reduce uncertainties in basal boundary conditions. 

Some of the latest technology in aerogeophysical surveys of Earth’s
	major ice sheets includes focused synthetic aperture radar
	(SAR) techniques. Focused radar products provide an even
	better picture of glacial isochrones by including additional
	clutter reduction and preserving echoes from sloped
	layers. This enhances our ability to detect and track layers,
	even in more complex englacial environments. [source – Peters
	paper] 

[More about the wonders of radar] 

\section{Statement of the problem [3 pages] }

With recent improvements in the processing of airborne radar sounding
observations, detection of internal layers has become increasingly
reliable and informative. Airborne radar surveys are being used to
study a wide range of glacial properties and processes such as glacial
hydrology, ice dynamics, and mass balance. Despite the advancement in
these observational techniques, however, there has been no
comprehensive quantification of the uncertainties associated with
airborne radio echo sounding. In some cases, such as near the base of
the ice sheet where opportunities for observations are limited, large
uncertainties can be expected. 

As with all data collection, it is important to recognize the
limitations of observational techniques. One way in which to do this
quantitatively is to do a thorough uncertainty analysis. This can
provide a more true picture of how much data can be trusted to present
an accurate picture of the physical state of a glacier. A
comprehensive review of uncertainties also offers an opportunity for
uncertainty reduction through refinement of observational techniques
by quantifying the contribution to uncertainty made by each aspect of
the observational method.

In addition to encouraging improved approaches to data collection,
uncertainty quantification is key to interpreting scientific
results. In the case of internal radar reflection horizons and ice
dynamics, there is uncertainty attributed to the determination of the
horizon depths as well as the ages assigned to those depths. The next
two section describe in more detail what the sources of uncertainty
are in each of these cases.

\subsection{Uncertainty in Two Way Travel Time(TWTT)}

	There is uncertainty in the picked depth of radar layers.
	GeoFrame, seismic interpretation software, allows a user to
	select strong reflectors from a radargram and trace them along
	a radar line. However, there is a fundamental limit to how
	accurate the depth of these layers be given the resolution of
	the sampling rate of the radar transmitter. This sampling rate
	reflects the fact that the radar operates by emitting
	electromagnetic pulses. The receiver then samples the
	reflections at a given rate. The sampling rate for the data
	used in this study varies from 5 ns to 20 ns and so we assume
	a resolution of 10 ns when picking the reflections from the
	surface of the ice sheet and from each internal layer. We
	treat the surface and internal layers separately in this
	instance. 

To convert TWTT uncertainty into units of depth, we scale the time by
	the velocity of electromagnetic wave propagation. For example,
	we assume the pulse travelled to the surface with velocity, c,
	the speed of light, but then slowed to the velocity of
	electromagnetic wave propagation in ice, which we assume to be
	1.69 x 10$^$8 m/s before reaching the internal layers. The
	corresponding 1$/sigma$ uncertainty is 0.3 m for the surface
	reflector and 0.17 m for the internal layers.  

 The finite bandwidth of the data means that even an infinitesimally
	thin layer of ice will appear in the survey to have a finite
	width. Our data used a pulse bandwidth of 15 MHz. This
	translates to a 1$\sigma$ depth uncertainty of 5.63 m,
	obtained from considering both the bandwidth frequency and the
	velocity of electromagnetic radiation in ice. This uncertainty
	is applied as a random, normally-distributed error in the
	depth of each of our selected layers. 

\subsection{Uncertainty in Byrd ice core chronology}

\section{Research Plan [3 pages] }

\section{References} 

\bibliographystyle{apj} \bibliography{layers}
\end{document}


